\documentclass[a4paper,12pt]{article}
\usepackage[utf8]{inputenc}
\usepackage[T2A]{fontenc}     
\usepackage[russian]{babel}
\usepackage{array,booktabs,float}
\usepackage{geometry}
\geometry{margin=2cm}
\pagestyle{empty}

\begin{document}


% Упражнение 1 — Простая таблица

\begin{table}[H]
\centering
\caption{Упражнение 1 — Простая таблица}
\begin{tabular}{lcr}
\toprule
Левый & Центр & Правый \\
\midrule
кот  & мясо  & small \\
пёс  & кости & medium \\
конь & сено  & large \\
\bottomrule
\end{tabular}
\end{table}


% Упражнение 2 — Разное выравнивание

\begin{table}[H]
\centering
\caption{Упражнение 2 — Разные типы выравнивания}
\begin{tabular}{l c r}
\toprule
\multicolumn{1}{c}{Left (l)} & Center (c) & Right (r) \\
\midrule
1.234     & 1.234     & 1.234     \\
123.4     & 123.4     & 123.4     \\
12345.678 & 12345.678 & 12345.678 \\
\bottomrule
\end{tabular}
\end{table}


% Упражнение 3 — Недостаток элементов

\begin{table}[H]
\centering
\caption{Упражнение 3 — Недостаток элементов в строке}
\begin{tabular}{lcr}
\toprule
A & B & C \\
\midrule
x & y & {}  \\ % пустая ячейка вместо недостающей
\bottomrule
\end{tabular}
\end{table}

% Упражнение 4 — Избыток элементов

\begin{table}[H]
\centering
\caption{Упражнение 4 — Избыток элементов в строке}
\begin{tabular}{lcr}
\toprule
A & B & C \\
\midrule
x & y & z & f   \\ % исправленная версия (лишняя удалена)
\bottomrule
\end{tabular}
\end{table}

% Упражнение 5 — Использование \multicolumn

\begin{table}[H]
\centering
\caption{Упражнение 5 — Использование \texttt{\textbackslash multicolumn}}
\begin{tabular}{lcc}
\toprule
\multicolumn{1}{c}{Группа} & \multicolumn{1}{c}{Метрика A} & \multicolumn{1}{c}{Метрика B} \\
\midrule
Модель X & 0.81 & 0.74 \\
Модель Y & 0.85 & 0.77 \\
\cmidrule(lr){2-3}
Итого & \multicolumn{2}{c}{среднее = 0.79} \\
\bottomrule
\end{tabular}
\end{table}


\end{document}

