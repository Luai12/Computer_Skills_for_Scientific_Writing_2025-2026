\documentclass{beamer}
\usepackage[utf8]{inputenc}
\usepackage[T2A]{fontenc}
\usepackage[russian]{babel}
\usepackage{graphicx}
\usepackage{amsmath}
\usepackage{caption} 
\usepackage{booktabs} 
\usepackage{hyperref} 

\usetheme{Madrid}
\usecolortheme{beaver}

\title{Лабораторная работа №7 }
\subtitle{Презентация на базе Beamer}
\author{Дабван Л.М. }
\institute{Университет: рудн}
\date{\today}

\begin{document}

\begin{frame}
\titlepage
\end{frame}

\begin{frame}{Содержание}
\tableofcontents
\end{frame}



\section{Введение}
\begin{frame}{Что такое Beamer?}
\begin{block}{Определение}
Beamer — класс LaTeX для создания презентаций.
\end{block}



\begin{itemize}
\item Создание профессиональных слайдов
\item Поддержка математических формул
\item Полный контроль над дизайном
\end{itemize}
\end{frame}

\section{Основная структура}
\begin{frame}{Структура документа Beamer}
\begin{exampleblock}{Упрощенный пример}
{\tiny
\texttt{\textbackslash documentclass\{beamer\}}\\
\texttt{\textbackslash begin\{document\}}\\
\texttt{\textbackslash begin\{frame\}\{Заголовок\}}\\
\texttt{Содержание слайда}\\
\texttt{\textbackslash end\{frame\}}\\
\texttt{\textbackslash end\{document\}}
}
\end{exampleblock}
\vspace{0.3cm}

\begin{alertblock}{Важно}
Каждый слайд должен быть внутри окружения \texttt{frame}.
\end{alertblock}
\end{frame}





\section{Элементы управления}
\begin{frame}{Контроль отображения}
\begin{itemize}
\item Первый элемент
\pause
\item Второй элемент
\pause
\item Третий элемент
\end{itemize}

\pause

\begin{block}{Блок после паузы}
Объяснение: В этом слайде я использовал команду. Она
позволяет показывать элементы на слайде поочередно.
\end{block}
\end{frame}



\section{Математика}
\begin{frame}{Математические формулы}
Формула в тексте: $E = mc^2$

\vspace{0.5cm}

Отдельная формула:
\[
\int_{0}^{1} x^2 \, dx = \frac{1}{3}
\]

\vspace{0.5cm}

Нумерованная формула:
\begin{equation}
\frac{d}{dx}(x^n) = n x^{n-1}
\end{equation}
\end{frame}

\begin{frame}{Выровненные уравнения}
\begin{align*}
f(x) &= x^2 + 2x + 1 \\
g(x) &= \sin(x) + \cos(x) \\
h(x) &= \frac{1}{1 + e^{-x}}
\end{align*}
\end{frame}

\section{Изображения и таблицы}
\begin{frame}{Изображения в презентациях}
\begin{center}
\includegraphics[width=0.5\textwidth]{image_01.jpg}
\captionof{figure}{Пример изображения}
\end{center}
\end{frame}



\begin{frame}{Таблицы в Beamer}
\begin{center}
\begin{tabular}{lcc}
\toprule
Предмет & Количество & Цена \\
\midrule
Книги & 5 & 2500 руб. \\
Ручки & 10 & 500 руб. \\
Блокноты & 3 & 900 руб. \\
\bottomrule
\end{tabular}
\end{center}
\end{frame}

\section{Советы}
\begin{frame}{Полезные советы}
\begin{enumerate}
\item Одна идея на слайд
\item Используйте изображения
\item Минимизируйте текст
\item Практикуйтесь
\item Следите за временем
\end{enumerate}
\end{frame}

\section{Заключение}
\begin{frame}{Выводы}
\begin{block}{Преимущества Beamer}
\begin{itemize}
\item Профессиональное качество
\item Отличная математическая поддержка
\item Кроссплатформенность
\end{itemize}
\end{block}

\begin{alertblock}{Начните сейчас}
Начните с простых слайдов и постепенно усложняйте.
\end{alertblock}
\end{frame}

\begin{frame}
\begin{center}
\Huge \textbf{Спасибо за внимание!}

\vspace{1cm}
\Large Вопросы?
\end{center}
\end{frame}

\end{document}

